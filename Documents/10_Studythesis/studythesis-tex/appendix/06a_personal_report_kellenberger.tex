\kapitel{Personal Report}
\thispagestyle{plain}
\renewcommand\section{\stdsection}
\sectionroman{Kellenberger Lukas}
For me, the study thesis was my biggest software project to date and after the engineering project also only my second one. The theoretical knowledge was given to me in the modules Software Engineering 1 and Software Engineering 2 and I was looking forward to applying what I had learned in practice. 
\\\\
At the beginning of the semester there were many unanswered questions and I did not know what expectations were placed on us. After a first meeting with our supervisor Cyrill Brunnschwiler many questions were answered, but as many new ones came up. As a first step we should familiarize ourselves with advanced persistent threats and lateral movement. We were both still new in this area and so we spent the first few weeks familiarizing ourselves with this topic. After familiarizing ourselves with the topic, we felt ready to define the scope of the study thesis. Our goal was to develope a tool by the end of the semester with which the readiness of a system could be displayed.
\\\\
We chose PowerShell to implement the project because of its proximity to the system. This decision was a bit hasty. In retrospect, it would have been smart to take a closer look at some other options, for example C\#. In the beginning, we got along well with PowerShell and were faster than expected. The longer the project lasted and the more complex it became, the more it turned out that Powershell was not the best possible solution. This was especially evident in the visualization of the results, where we needed much more time than expected.
\\\\
When writing the code I was sometimes a little too hasty, so I noticed only after implementing the GetEventLogsAndExport() function that the runtime is not practicable. I noticed that the runtime is O(n*m) and I could limit the runtime to O(n) with another variant. This mistake easily costed me a few hours. With such mistakes I notice that I still lack some experience, but I am sure that I learn from these mistakes.
\\\\
I felt that the teamwork and communication within the team was very good, we also organised ourselves in such a way that we could often work together. However, it was precisely for this reason that we did not stick to Scrum how we should have done it. Also the code reviews were neglected with the time which then became noticeable with the refactoring. This is something we need to pay more attention to in the next project.
\\\\
On the whole, we were always able to estimate the time required for the work well, but we were far off the mark when it came to visualization. The fact that we booked a little more time than we thought is more due to our own interest and will than to false estimates. One point where I still have a lot of room for improvement is the booking of times, which i have often forgotten.
\\\\
All in all, I am very happy with our study thesis. We worked well together as a team and found a suitable solution for all problems. I was able to apply a lot of what I had already learned and have gained many valuable new experiences. I am really looking forward to continuing this work as a Bachelor thesis and I hope to be able to complete this project as successfully as this one.
