\sectionroman{Mattes Claudio}
I approached the study thesis with the expectation of more from the implementation of a software project as well as the deepening of learned theories, programming skills and methods learned during the study. Also for me, it was another new experience to participate in a software project. The security area of computer science had already aroused my fascination a long time ago. As a relatively inexperienced software developer, combining the study thesis in the areas of security and software development motivated me tremendously to tackle this thesis.
\\\\
Due to the very open nature of the task, it was difficult for me to find my way into the work at the beginning and not to lose track of it. To find my way into a new topic, in which I was not yet really familiar, turned out to be an exciting challenge. After some meetings with our supervisor Cyrill Brunnschwiler many question marks could be eliminated and the way to go became clearer and clearer.
\\\\
The chosen programming language PowerShell turned out to be suitable but with restrictions. An almost complete rethinking of the software development had to be made, since almost only an object-oriented approach in the classical sense of software development is taught during the studies. During the study thesis, I realized that the approach to handle the task with a script-based language has some limitations. Especially to build a suitable main script for the user turned out to be a bigger challenge than expected. Without a visual user interface, it was not easy to provide a user-friendly environment for the end user. Also the test framework did not offer the functionality learned in the software modules. In retrospect, I would put more focus and effort on the chosen technologies.
\\\\
Looking back, I consider that the project management and the Scrum method can be applied even better. Especially in the next project, I would like to make sure that the sprint planning is carried out more consistently. The impact was that sometimes the project management tool was not completely cleaned up.
\\\\
All in all, I found the team dynamics as well as the communication in the team very productive and relaxed. Everyone could contribute their opinions and ideas and was respected by the team member. It was really fun to do this project, even though there were some frustrating moments while developing - but they could be quickly resolved with the help of the team mate. 
\\\\
I am pleased to continue this study thesis as a bachelor thesis with my partner Lukas Kellenberger and our supervisor Cyrill Brunnschwiler. I am looking forward to gain further experience in software development and security.
\\\\
At this point I would like to thank Lukas Kellenberger for the constructive and open cooperation. A thank you also goes to our supervisor Cyrill Brunnschwiler, who always showed us the right direction during the project.
