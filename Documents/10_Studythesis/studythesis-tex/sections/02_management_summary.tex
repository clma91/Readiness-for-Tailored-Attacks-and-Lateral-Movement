\kapitel{Management Summary}
\addcontentsline{toc}{part}{Management Summary}
\thispagestyle{plain}
\renewcommand\section{\stdsection}
\vspace{-0.25cm}
\sectionroman{Initial Situation}
\thispagestyle{plain}
The number of cyber-attacks where malicious code is used, which not only settles on the infected system, but also infects other systems in the network, has massively increased recently. The outcome is often the complete infiltration of the organization. In the analysis of such an event, information and time are key factors to success. Consequently, readiness for such an event is a decisive factor. \\\\
The ''Japan Computer Emergency Response Team Coordination Center'' has analysed the procedure and the used tools of such attacks. In their most recent publication on this topic, they give hints which events indicate a possible contamination. The aim of this study thesis is to use this published paper and write a PoC that helps to identify the readiness of a system. The readiness of a system indicates whether solid conclusions can be drawn about the attacker in the event of an attack.
\vspace{-0.25cm}
\sectionroman{Procedure}
\thispagestyle{plain}
The project was initially limited to Windows machines running on the operating system Windows 10 Pro or Windows Server 2016. The project was handled according to common project management and software engineering principles. During the elaboration phase we did some research on the topic and we tested different tools which cloud be interesting for our project. At the end of the elaboration we had decided to realise the project using PowerShell. In the following six weeks we wrote a PoC - the ''System Readiness Inspector - SRI''. The SRI reads information about the system on which it is running and evaluates which attack categories (e.g. command execution, password hash acquisition, deleting evidence on a system and so on) can or cannot be detected with these settings. This information obtained is then visualised into a PDF document and output by the script.
\vspace{-0.25cm}
\sectionroman{Results}
\thispagestyle{plain}
The SRI runs successfully and outputs important system settings about the readiness. Illustrated in a PDF, the analyst can see at a glance which of his audit settings are missing or incorrect. The script also evaluates which attacks might be missed due to incorrectly configured settings. SRI helps an analyst to check a system for its readiness and saves him the tedious task of collecting and evaluating the data.
\vspace{-0.25cm}
\sectionroman{Outlook}
\thispagestyle{plain}
SRI is still at an early stage of its development. The further development of the visualisation is conceivable. The extension to an entire fleet will also be an approach that will certainly be pursued further. Nevertheless, SRI is a useful helper when it comes to get a quick overview about the audit settings and the readiness in general.



\renewcommand\section{\clearpage\stdsection}