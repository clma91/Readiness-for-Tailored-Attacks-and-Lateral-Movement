
\kapitel{General Information} \label{GeneralInfo}
\thispagestyle{plain}
\renewcommand\section{\stdsection}
\setcounter{section}{1}
\subsection{System Overview}
The ''System Readiness Inspector'' is a PowerShell tool that helps you to check the readiness of a system to detect advanced persistent threats and lateral movement.After the SRI ran successfully it generates a PDF-Document showing wrong or missung configurations. The SRI was developed during a student research project by the two bachelor of science in computer science students, Claudio Mattes and Lukas Kellenberger.
\\\\
The SRI has four different modes: Online, Offline, GroupPolicy, AllGroupPolicies. The online mode is limited to the current system and thus determines readiness. The offline mode is used to be able to make a statement about any system by means of exports.
The GroupPolicy mode is limited to a specific Group Policy, which is checked for its
audit settings. In the AllGroupPolicies mode, all group policies of the current domain are examined.

\subsection{Organization of the Manual}
The user manual consists of five parts: 
\begin{itemize}
    \item \textbf{General Information:} \ \\
The General Information section explains the tool and the purpose for which it is intended.
    \item \textbf{System Requirements: } \ \\
The System Requirements section provides a general overview of the system requirements. Which operating systems are supported, what software must be pre-installed, and what authorizations the user must have.
    \item \textbf{Getting Started: } \ \\
The Getting Started section explains how to obtain and install the SRI on your device. 
    \item \textbf{Using SRI: }\ \\
The Use SRI section provides a detailed description of the system functions. 
\end{itemize}
