
\section{Managementabläufe}
\subsection{Kostenvoranschlag}
Das Projekt wird im Herbstsemester 2018 umgesetzt. Dieses dauert vom 17.09.2018 bis am 21.12.2018. Dabei stehen 14 Semesterwochen zur Verfügung. Pro Woche wird mit einem Zeitaufwand von 34 Stunden\footnote{ 17 Stunden pro Person} pro Woche gerechnet. Dies ergibt rund 480 Stunden über die Gesamtdauer des Projekts. Für Sitzungen sind pro Woche ca. 2 Stunden\footnote{ 1 Stunde pro Woche mit dem Betreuer, sowie 1 Stunde für Iterations- und Teaminterne-Planungen} eingeplant. 

\subsection{Zeitliche Planung}
Grundsätzlich wird während dieser Arbeit nach der Vorgehensweise Scrum+ (Scrum vereint mit RUP) gearbeitet. Das Projekt wird in vier Phasen Inception, Elaboration, Construction und Transition aufgeteilt. Wobei der Ansatz Scrum ausschlieslich in der Construction-Phase zur Anwendung kommen wird, mit Iterationslängen von 2 Wochen.

\subsubsection{Phasen}
\begin{table}[H]
    \centering
    \def\arraystretch{2}
    \begin{tabular}{| p{2.5cm} | p{10.5cm} | p{3cm} |} \hline
        \textbf{Phase} & \textbf{Beschreibung} & \textbf{Dauer} \\ \hline
        Inception & Kickoff Meeting, Aufgabestellung, Einarbeitung & 1 Woche \\ \hline
        Elaboration & Projektplan, Meilensteindefinition, Anforderungen, Research & 5 Wochen \\ \hline
        Construction & Realisierung des Toolkits, Testing, Dokumentation & 6 Wochen \\ \hline 
        Transition & Abschluss Dokumentation, Projektreflexion, Poster & 2 Wochen \\ \hline
    \end{tabular}
    \caption{Projektphasen und deren Inhalte}
\end{table}

\subsubsection{Meilensteinübersicht}
\begin{figure}[H]
    \centering
    \includegraphics[width=1\linewidth]{assets/meilensteinuebersicht}
    \caption{Meilensteinübersicht}
\end{figure}
\clearpage

\subsubsection{Meilensteine}
\begin{table}[H]
    \centering
    \def\arraystretch{2}
    \begin{tabular}{|l|c|p{0.55\linewidth}|} \hline
        \textbf{Name} & \textbf{Datum} & \textbf{Work Products} \\ \hline
        M0 – Kickoff Meeting & 19.09.2018 &
        \vspace{-7mm}
        \begin{itemize}
            \setlength\itemsep{0mm}
            \item Unterzeichnung Aufgabestellung
            \item Entgegennahme/Besprechung Aufgabestellung
            \vspace{-\topsep}
        \end{itemize} \\ \hline
        M1 – Projektplan & 02.10.2018 &
        \vspace{-7mm}
        \begin{itemize}
            \setlength\itemsep{0mm}
            \item Projektplan in erster Version fertig
            \vspace{-\topsep}
        \end{itemize} \\ \hline
        M2 – End of Elaboration & 23.10.2018 &
        \vspace{-7mm}
        \begin{itemize}
            \setlength\itemsep{0mm}
            \item Abschluss Research
            \item Continuous Integration aufgesetzt
            \item Use Cases definiert
            \item Scope Definition des Toolkits
            \vspace{-\topsep}
        \end{itemize} \\ \hline
        M3 – Zwischenbericht & 06.11.2018 &
        \vspace{-7mm}
        \begin{itemize}
            \setlength\itemsep{0mm}
            \item Zwischenbericht mit aktuellem Stand und allfälligen Zielanpassungen
            \vspace{-\topsep}
        \end{itemize} \\ \hline
        M4 – Feature Freeze & 27.11.2018 &
        \vspace{-7mm}
        \begin{itemize}
            \setlength\itemsep{0mm}
            \item Prototyp mit wichtigste Funktionalitäten erstellt
            \vspace{-\topsep}
        \end{itemize} \\ \hline
        M5 – End of Construction & 4.12.2018 &
        \vspace{-7mm}
        \begin{itemize}
            \setlength\itemsep{0mm}
            \item Alle geplanten Features sind implementiert und getestet \& Bugfixes gemacht
            \vspace{-\topsep}
        \end{itemize} \\ \hline
        M6 – Abgabe & 21.12.2018 & 
        \vspace{-7mm}
        \begin{itemize}
            \setlength\itemsep{0mm}
            \item Abgabe aller Dokumente
            \vspace{-\topsep}
        \end{itemize} \\ \hline

    \end{tabular}
    \caption{Meilensteine}
\end{table}


\subsection{Meetings}
\paragraph{Projektmeeting}
Die Besprechungen mit dem Betreuer finden in der Regel Dienstagvormittag von 08:30 bis 09:30 Uhr statt. Als Ausweichtermin gilt der Donnerstagnachmittag von 13:00 bis 14:00 Uhr.
\paragraph{Teammeeting}
Das Team trifft sich jeweils Freitagnachmittags zu Besprechungen oder gemeinsamen Reviews.