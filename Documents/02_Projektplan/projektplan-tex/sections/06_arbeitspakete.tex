\section{Arbeitspakete}
Die detaillierte Ansicht der Arbeitspakete ist im Redmine ersichtlich. Der Zugriff erfolgt über folgende URL:
\begin{itemize}
    \item \url{http://sinv-56085.edu.hsr.ch/redmine/projects/readiness-for-tailored-attacks-and-lateral-movement-detection/issues}
\end{itemize}


\subsection{Aufwand und Schätzung}
Der Aufwand wird in Arbeitsstunden angegeben. Diese Stunden sind Schätzungen und können bei auftretenden Problemen oder schnelleren Ausführungen abweichen.

\subsection{Priorisierung}
Die Arbeitspakete werden wie folgt priorisiert:
\begin{description}
    \item[Niedrig] Feature ist optional, erfolgreicher Projektabschluss hängt nicht von diesem Arbeitspaket ab.
    \item[Normal] Feature wird im Projekt benötigt, andere Arbeitspakete setzen dieses Feature voraus
    \item[Hoch] Feature ist zwingend Notwendig für einen erfolgreichen    Projektabschluss
\end{description}
Wir fokussieren uns im Projekt auf die Realisierung aller Arbeitspakete, welche die Priorität Hoch sowie Normal erhalten haben. Arbeitspakete welche die Priorität Niedrig erhalten haben, werden bei vorhandener Zeit implementiert. Sollte keine Zeit mehr vorhanden sein, wird dieses Feature mit dem nächsten Release ausgerollt.
