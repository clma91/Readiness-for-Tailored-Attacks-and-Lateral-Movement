\section{Risikomanagement}
\subsection{Risiken}
\subsubsection{R1 - Code Datenverlust}
\begin{table}[H]
    \centering
    \def\arraystretch{2}
    \begin{tabular}{| p{4.5cm} | p{13.5cm} |} \hline
        \textbf{Beschreibung} & Der Code auf Git geht verloren \\ \hline
        \textbf{Massnahme} & Anlegen von lokalen Arbeitskopien  \\ \hline
        \textbf{Verhalten bei Eintritt} & Lokale Arbeitskopien der Teammitglieder werden zusammengeführt und ins Git Repository hochgeladen \\ \hline 
    \end{tabular}
    \caption{R1 - Code Datenverlust}
\end{table}

\subsubsection{R2 - Dokumentverlust}
\begin{table}[H]
    \centering
    \def\arraystretch{2}
    \begin{tabular}{| p{4.5cm} | p{13.5cm} |} \hline
        \textbf{Beschreibung} & Die Dokumente auf OneDrive gehen verloren \\ \hline
        \textbf{Massnahme} & Wöchentlich lokale Kopien und Versionierung der Dokumente  \\ \hline
        \textbf{Verhalten bei Eintritt} & Zurückspielen der lokalen Datensicherung \\ \hline 
    \end{tabular}
    \caption{R2 - Dokumentverlust}
\end{table}

\subsubsection{R3 - Ausfall Redmine}
\begin{table}[H]
    \centering
    \def\arraystretch{2}
    \begin{tabular}{| p{4.5cm} | p{13.5cm} |} \hline
        \textbf{Beschreibung} & Redmine ist nicht erreichbar\\ \hline
        \textbf{Massnahme} & Keine Experimente auf dem System mit der Redmine-Installation  \\ \hline
        \textbf{Verhalten bei Eintritt} & Server neustarten, Problem analysieren und innert nützlicher Frist behebeng \\ \hline 
    \end{tabular}
    \caption{R3 - Ausfall Redmine}
\end{table}

\subsubsection{R4 - Datenverlust Redmine}
\begin{table}[H]
    \centering
    \def\arraystretch{2}
    \begin{tabular}{| p{4.5cm} | p{13.5cm} |} \hline
        \textbf{Beschreibung} & Redmine-Datenbank beschädigt oder inkonsistent\\ \hline
        \textbf{Massnahme} & Wöchentliche CSV-Exports aller wichtigen Redmine-Komponenten  \\ \hline
        \textbf{Verhalten bei Eintritt} & Je nach Fehler den Server neu aufsetzen, Redmine neu installieren, neusten CSV-Export in Redmine importieren \\ \hline 
    \end{tabular}
    \caption{R4 - Datenverlust Redmine}
\end{table}

\subsubsection{R5 - Ungenügende Kommunikation}
\begin{table}[H]
    \centering
    \def\arraystretch{2}
    \begin{tabular}{| p{4.5cm} | p{13.5cm} |} \hline
        \textbf{Beschreibung} & Kommunikation mit Kunde /Betreuer erweist sich als ungenügend oder schwierig \\ \hline
        \textbf{Massnahme} & Regelmässige Planung von Sitzungen\\ \hline
        \textbf{Verhalten bei Eintritt} & Planung der Sitzungen überarbeiten, Kontakt mit Kunde / Betreuer aufnehmen \\ \hline 
    \end{tabular}
    \caption{R5 - Ungenügende Kommunikation}
\end{table}

\subsubsection{R6 - Unterschätzte Komplexität}
\begin{table}[H]
    \centering
    \def\arraystretch{2}
    \begin{tabular}{| p{4.5cm} | p{13.5cm} |} \hline
        \textbf{Beschreibung} & Kommunikation mit Kunde /Betreuer erweist sich als ungenügend oder schwierig \\ \hline
        \textbf{Massnahme} & Regelmässige Planung von Sitzungen\\ \hline
        \textbf{Verhalten bei Eintritt} & Planung der Sitzungen überarbeiten, Kontakt mit Kunde / Betreuer aufnehmen \\ \hline 
    \end{tabular}
    \caption{R6 - Unterschätzte Komplexität}
\end{table}

\subsubsection{R7 - Ausfall von Entwickler-PC}
\begin{table}[H]
    \centering
    \def\arraystretch{2}
    \begin{tabular}{| p{4.5cm} | p{13.5cm} |} \hline
        \textbf{Beschreibung} & PC eines Teammitglieds funktioniert nicht mehr und verhindert die Weiterarbeit am Projekt \\ \hline
        \textbf{Massnahme} & Versionsverwaltungs-Werkzeug, Dokumentationen für die Installation wichtiger Komponenten, allgemeine Regeln im Umgang mit PCs beachten\\ \hline
        \textbf{Verhalten bei Eintritt} & PC neu installieren oder Ersatz besorgen, Entwicklerumgebung gemäss Dokumentation einrichten und alle aktuellen Versionen des Projektes ins System importieren \\ \hline 
    \end{tabular}
    \caption{R7 - Ausfall von Entwickler-PC}
\end{table}

\subsection{Umgang mit Risiken}
Für jedes identifizierte Risiko in der Risikoplanung wird am Anfang versucht durch geeignete Massnahmen das jeweilige Risiko zu minimieren.
