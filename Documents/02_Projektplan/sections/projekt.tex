\section{Projekt Übersicht}

\subsection{Ziel und Zweck}
Es werden vermehrt Cyberangriffe publik, wo Schadcode im Einsatz ist, welcher sich nicht nur auf einem infizierten System niederlässt, sondern weitere Systeme im Netz befällt. Das Ziel oder Resultat ist dabei oft die komplette Infiltrierung einer Organisation. In der Analyse solcher Fälle sind Information und Zeit ein Schlüssel zum Erfolg. Folglich ist die Bereitschaft ''Readiness'' für ein solches Ereignis ein entscheidender Faktor.
\\\\
Ziel dieser Arbeit ist es, ein Tool zu erstellen, welches die Bewertung der eigene Readiness erlaubt aber auch im Analysefall eine Unterstützung bietet. Readiness betrifft viele Aspekte und einfache Dinge wie korrekte Zeitstempel in Logs, deren Vollständigkeit oder die Bereitstellung von Backups. In der konkreten Aufgabenstellung soll die Readiness-Analyse primär für Windows-Infrastrukturen anhand von Logs und spezifischen Events erfolgen. Unter anderem soll auf den neusten Publikationen des japanischen Computer Emergency Response Teams (JPCERT/CC) und der öffentlichen Datenbank der MITRE Corporation, dem Adversarial Tactics, Techniques, and Common Knowledge (ATT\&CK™) Wissenspool, basiert werden. Das JPCERT und MITRE haben dabei die Werkzeuge und das generelle Vorgehen von Angreifern analysiert und geben Hinweise, welche Events auf eine mögliche Verseuchung hinweisen.

\subsection{Vorgehen}
Das Vorgehen dieser Arbeit ist in einem ersten Schritt eine gründliche Einarbeitung in die Themen
\begin{itemize}
    \item Incident Handling und Forensik
    \item Angriffstechniken und Werkzeuge
    \item Abwehrtechniken und Härtung von Systemen
\end{itemize}
sowie das Studium öffentlicher Quellen und verfügbarer Tools. Während dieses Prozesses werden noch unbekannte Fragen, wie beispielsweise auf welchen Tools aufgebaut und in welchem Ausmass diese erweitert werden, geklärt. Zudem soll der genau Scope der zu erreichenden Funktionalitäten des Toolkits konkret definiert werden.


\subsection{Lieferumfang}
\begin{itemize}
    \item lauffähiges Toolkit und kompletter Source Code
    \item komplette Software Dokumentation 
    \item komplette Use Cases und Erfolgs-Szenarien resp. Musterlösungen
\end{itemize}
Die Abgabe des Berichtes erfolgt als gebundenes Dokument. Alle erarbeiteten Daten werden zusätzlich in elektronischer Form auf archiv-i.hsr.ch bereitgestellt.

\subsection{Annahmen und Einschränkungen}
Es gehört nicht zum Umfang dieser Arbeit neue Angriffsvektoren zu finden. Da dieses Projekt in Form eines Modules mit 8 ECTS Punkten durchgeführt wird, soll dieses mit 2 beteiligten Personen im Rahmen von 480 Arbeitsstunden durchgeführt werden. Das Projekt endet dabei aber spätestens am letzten Studientag es Herbstsemester dem 21. Dezember 2018.