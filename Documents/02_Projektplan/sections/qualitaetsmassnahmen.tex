\section{Qualitätsmassnamen}
In diesem Kapitel wird definiert, mit welchen Massnahmen die Qualität des Projekts so hoch wie möglich gehalten wird.

\subsection{Dokumentation}
Für die Dokumentation besteht ein eigenes Git-Repository. Die Dokumente unterstehen dem Vier-Augen-Prinzip. Dadurch werden Fehler minimiert und die Qualität entsprechend gesteigert.

\subsection{Research Dokumente und Tools}
Alle verwendeten Dokumente während der Research-Phase werden in einer OneDrive Ablage gehalten. Für die jeweiligen Dokumente und Tools wird jeweils in einer Tabelle eine kurze Zusammenfassung, sowie eine Bewertung der Relevanz für die Studienarbeit aufgeführt.

\subsection{Projektmanagement}
Als Projektmanagement-Tool haben wir uns für Redmine entschieden, welches sich auf dem virtuellen Server der HSR befindet.

\subsubsection{Roadmap}
Damit jederzeit die Iteration im Blick gehalten werden kann, bietet Redmine eine Roadmap an. Dort sieht man sofort, wann und welche Iteration erreicht werden muss, sowie auch welche Features diese Iteration beinhaltet und den Status der Features.
\\\\
Einen weiteren guten Überblick erhält man mit dem in Redmine integrierten Gant-Diagramm.

\subsection{Definition of Done}
Arbeitspakete werden als abgeschlossen betrachtet, sofern sie folgende Kriterien erfüllen:
\begin{itemize}
    \item Funktionalität gemäss Beschreibung implementiert
    \item Feature wurde nachgeführt und geschlossen
    \item Allfällige Dokumentationen wurden angepasst
    \item Nachführung von Informationen und Zeiterfassung erledigt
    \item Issue wurde geschlossen
\end{itemize}

\subsection{Entwicklung}
Der Source-Code befindet sich in einem Git-Repository von GitHub, und kann dank Versionierung getrackt werden. Die Qualität wird durch regelmässige Code Reviews und durch das verwenden einer Code Style
Guideline sichergestellt.

\subsubsection{Unit Testing}
Für alle wichtigen Klassen und Komponenten werden systematisch und regelmässig Unit Tests geschrieben. Diese Tests sollen garantieren das der Code fehlerfrei läuft.

\subsubsection{Coding Guidelines}
\begin{itemize}
    \item Kein unaufgeräumter (z.B. auskommentierter) Code
    \item Alles Unfertige ist markiert mit «TODO:»
    \item Code ist wo nötig \& richtig mit Kommentaren versehen
\end{itemize}

\subsubsection{Code Reviews}
Die Teammitglieder reflektieren den Code der Anderen in regelmässigen Abständen und geben Verbesserungsvorschläge, welche im Team ausdiskutiert und umgesetzt werden können.

\subsubsection{Pair Programming}
Je nach Zeitreserven finden auch Pair Programming Sessions statt. Dabei programmiert einer und erläutert laut seine Gedankengänge und der zweite verfolgt die Gedankengänge und Codezeilen, die der andere programmiert. Er meldet sich jeweils, wenn er einen anderen Lösungsweg hat, einen Fehler   entdeckt oder auch wenn alles in Ordnung ist.