\newcommand{\TITLE}{Readiness for Tailored Attacks and Lateral Movement Detection}
\newcommand{\REVIEW}{Weekly Meeting}
\newcommand{\VERSION}{0.0}

\documentclass[a4paper, oneside, 11pt]{report}
    \usepackage[T1]{fontenc}
    \usepackage[utf8]{inputenc}
    \usepackage[nswissgerman, english]{babel}
    \usepackage[hyphens]{url}
    \usepackage[hidelinks]{hyperref}
    \usepackage{graphicx}
    \usepackage{subfig}
    \usepackage{vhistory}
    \usepackage{float}
    \usepackage{pdfpages}
    \usepackage{tcolorbox}
    \usepackage{xcolor}
    \usepackage{nameref}
    
    % Seitenränder
    \usepackage{geometry}
    \geometry{
        a4paper,
        left=20mm,
        top=30mm,
        right=20mm,
        bottom=30mm
    }

    % Jede Überschrift 1 auf neuer Seite
    \let\stdsection\section
    \renewcommand\section{\clearpage\stdsection}

    % Header and footer
    \usepackage{fancyhdr}
    \pagestyle{fancy}
    \fancyhf{}
    \lhead{\small \TITLE \\ \vspace{0.5mm} \normalsize \nouppercase\rightmark \vspace{0.0cm}}
    \rhead{
        \begin{picture}
            (0,0) \put(-100,0){\includegraphics[width=0.2\linewidth]{./assets/logo/hsr.png}}
        \end{picture}}
    \cfoot{\thepage}

    % Multicolomns
    \usepackage{multicol}
    \setlength{\multicolsep}{2.0pt plus 2.0pt minus 1.5pt}% 50% of original values (above/below multicols)

    % Chapter ohne Nummerierung, Eintrag in Inhaltsverzeichnis
    \newcommand{\kapitel}[1]{
        \stepcounter{chapter}\chapter*{#1}
        \addcontentsline{toc}{chapter}{#1}
        \markboth{\arabic{chapter} #1}{\arabic{chapter} #1}
    }
    % Section ohne Nummerierung, Eintrag in Inhaltsverzeichnis
    \newcommand{\sectionroman}[1]{
        \stepcounter{section}\section*{#1}
        \addcontentsline{toc}{section}{#1}
        \markboth{\arabic{section} #1}{\arabic{section} #1}
    }

    % Anpassung der Inhaltsverzeichnis-Tiefe, beginnend bei section
    \renewcommand{\partname}{}
    \renewcommand{\thesection}{\arabic{section}}
    \setcounter{secnumdepth}{3}
    \setcounter{tocdepth}{3}

    % Code Listings
    \usepackage{listings}
    \usepackage{color}
    \usepackage{beramono}

    \definecolor{bluekeywords}{rgb}{0,0,1}
    \definecolor{greencomments}{rgb}{0,0.5,0}
    \definecolor{redstrings}{rgb}{0.64,0.08,0.08}
    \definecolor{xmlcomments}{rgb}{0.5,0.5,0.5}
    \definecolor{types}{rgb}{0.17,0.57,0.68}

    \lstdefinestyle{visual-studio-style}{
        language=[Sharp]C,
        columns=flexible,
        showstringspaces=false,
        basicstyle=\footnotesize\ttfamily, 
        commentstyle=\color{greencomments},
        morekeywords={partial, var, value, get, set},
        keywordstyle=\bfseries\color{bluekeywords},
        stringstyle=\color{redstrings},
        breaklines=true,
        breakatwhitespace=true,
        tabsize=4,
        numbers=left,
        numberstyle=\tiny\color{black},
        frame=lines,
        showspaces=false,
        showtabs=false,
        escapeinside={£}{£},
    }
    \lstset{style=visual-studio-style}


\begin{document}

\section*{Sitzungsprotokoll}

\begin{tabular}{p{4cm} p{12cm}}
    \textbf{Projekt:} & Readiness for Tailored Attacks and Lateral Movement  \\
    \textbf{Woche:} & 6 \\
    \textbf{Datum / Zeit} & 23.10.2018 07:30 - 08:30 \\
\end{tabular}

\subsection*{Sitzungsteilnehmer / E-Mail}
\begin{table}[H]
    \centering
    \begin{tabular}{p{4cm} p{12cm}} \hline
        Claudio Mattes & claudio.mattes@hsr.ch \\ \hline
        Cyrill Brunschwiler & cyrill.brunschwiler@hsr.ch \\ \hline
    \end{tabular}
\end{table}

\vspace{1cm}

\subsection*{Traktanden}
\begin{itemize}
    \item Stand des Projekts
    \item Fragen
    \item Weiteres Vorgehen
\end{itemize}

\clearpage


\subsection*{Stand des Projekts}
\begin{table}[H]
    \centering
    \begin{tabular}{p{12cm} p{4cm}}
        \textbf{Arbeiten} & \textbf{Status} \\ \hline
        Research und Übersicht schaffen & Abgeschlossen \\ \hline
        Definition von Use Cases und NFRs & Abgeschlossen\\ \hline
        Schreiben der Dokumentation & in Bearbeitung \\ \hline
        Einarbeitung Powershell, einzelne Funktionen schreiben/testen & in Bearbeitung 
        \\ \hline
        PoC Sysmon-Detektion & Erreicht \\ \hline
    \end{tabular}
\end{table}

\vspace{1cm}

\subsection*{Unterstützungen}
\begin{table}[H]
    \centering
    \begin{tabular}{p{12cm} p{4cm}}
        \textbf{Art der Unterstützung} & \textbf{Hilfsperson} \\ \hline
        Keine & - \\ \hline
    \end{tabular}
\end{table}

\vspace{1cm}

\subsection*{Fragen}
\begin{itemize}
    \item Cobalt Strike
    \item Domain Model (unserer Meinung nach lohnt es sich nicht ein objektorientierte Lösung anzustreben, da es mehr Boilerplate wäre)
\end{itemize}

\vspace{1cm}

\subsection*{Weiteres Vorgehen}
\begin{table}[H]
    \centering
    \begin{tabular}{p{12cm} p{4cm}}
        \textbf{Was} & \textbf{Verantwortlichkeit} \\ \hline
        Domain Model Ausarbeitung (Ist-Zustand) & Team \\ \hline
        PoC Audit Policy & Claudio \\ \hline
        PoC Event Log & Lukas \\ \hline
        Dokumentation: Analyse fertigstellen & Team \\ \hline
        Dokumentation: Technologie fertigstellen & Team \\ \hline
        Ausarbeitung und Dokumentation Design des Tools & Team \\ \hline
        Beginn Implementation & Team \\ \hline
    \end{tabular}
\end{table}

\clearpage

\subsubsection*{Nächster Termin}

\begin{tabular}{p{4cm} p{12cm}}
    Datum: & 31.10.2018 \\
    Zeit: & 08:30 - 09:30 \\
    Ort: & Standort Alice \\
\end{tabular}

\vspace{1cm}

\subsection*{Kommende Abwesenheiten}
\begin{table}[H]
    \centering
    \begin{tabular}{p{6cm} p{5cm} p{5cm}}
        \textbf{Person} & \textbf{Von} & \textbf{Bis} \\ \hline
        Lukas Kellenberger & 26.10.18 & 29.10.18 \\ \hline
    \end{tabular}
\end{table}

\vspace{1cm}

\subsection*{Beschlüsse (Diskussion)}
\begin{itemize}
    \item Design des Tools
    \item Neue Risikobeurteilung (Krankheit, Verzug)
    \item Cobalt Strike wird keine Option sein
    \item Sysmon-Tools (olafhartong/sysmon-modular, nshalabi/SysmonTools) zur Prävention und korrekter Konfiguration von Sysmon in Analyse aufnehmen
    \item Transaction-Safety (was geschieht mit den Files, wenn das Tool abstürzt)
    \item Domain Model zur Zeit im Soll-Zustand $\rightarrow$ Ist-Zustand mit Client/Server und Verify/Analyse-Prozess aufnehmen
    \begin{itemize}
        \item Für zukünftige Weiterentwicklung und richtige Herangehensweise
    \end{itemize}
    \item Gibt es ein Pattern für das Domain Model?
    \item Objektorientierter Ansatz wird nicht verfolgt, da zu viel Boilerplate Code für das Tool entstehen sowie die Performanz darunter leiden würde
\end{itemize}

\end{document}